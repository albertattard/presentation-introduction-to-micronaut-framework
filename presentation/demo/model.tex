\section{Model}\label{sec:model}

Return a JSON object with more information for the endpoint \texttt{/greeting}, as shown next.

\begin{lstlisting}[language=bash]
$ curl http://localhost:8080/greeting/Albert
{"message":"Hello Albert, from Micronaut","time":"2020-04-27T20:00:00.000+02:00"}
\end{lstlisting}

\begin{enumerate}

\item Format date and time as String

By defult time are formatted as timestamps as shown next.

\begin{lstlisting}[language=bash]
{"message":"Hello Albert, from Micronaut","time":1588017600.000000000}
\end{lstlisting}

Setting \texttt{writeDatesAsTimestamps} to \texttt{false} will print dates and times as string.

File: \texttt{src/main/resources/application.yml}
\begin{lstlisting}[language=Kotlin]
jackson:
  serialization:
    writeDatesAsTimestamps: false
\end{lstlisting}

\item Create the \href{https://en.wikipedia.org/wiki/Data_transfer_object}{transfer object} to be returned.

File: \texttt{src/main/kotlin/demo/GreetResponse.kt}
\begin{lstlisting}[language=Kotlin]
package demo

import io.micronaut.core.annotation.Introspected
import java.time.ZonedDateTime

@Introspected
data class GreetResponse(
  val message: String,
  val time: ZonedDateTime
)
\end{lstlisting}

\href{https://docs.micronaut.io/2.0.0.M2/api/io/micronaut/core/annotation/Introspected.html}{\texttt{@Introspected}} indicates that the type should produce a \href{https://docs.micronaut.io/2.0.0.M2/api/io/micronaut/core/beans/BeanIntrospection.html}{\texttt{BeanIntrospection}} at \textbf{compilation time}.  Jackson will use this to format the \texttt{GreetResponse} to \texttt{JSON} without using reflection.

\item Update the test

File: \texttt{src/test/kotlin/demo/GreetingControllerTest.kt}
\begin{lstlisting}[language=Kotlin]
"should return the greeting message returned by the greeting service" {
  val name = "Albert"
  val result = client.toBlocking().retrieve("/$name$", GreetResponse::class.java)
  result.message shouldBe "Hello $name$, from Micronaut"
}
\end{lstlisting}

Run the tests

\begin{lstlisting}[language=bash]
$ ./gradlew clean build
\end{lstlisting}

The tests will fail

\begin{lstlisting}[language=bash]
...
> Task :test

demo.GreetingControllerTest > should return the greeting message returned by the greeting service FAILED
    io.micronaut.http.client.exceptions.HttpClientResponseException at DefaultHttpClient.java:2075
        Caused by: io.micronaut.http.codec.CodecException at TextPlainCodec.java:95

1 test completed, 1 failed

> Task :test FAILED

FAILURE: Build failed with an exception.

...
BUILD FAILED in 11s
18 actionable tasks: 18 executed
\end{lstlisting}

\item Update the controller

\begin{enumerate}

\item Change type media type

\begin{lstlisting}[language=Kotlin]
@Produces(MediaType.APPLICATION_JSON)
\end{lstlisting}

\item Change the function return type

\begin{lstlisting}[language=Kotlin]
fun greet(name: String): GreetResponse =
\end{lstlisting}

\item Return an instance of \texttt{GreetResponse}

\begin{lstlisting}[language=Kotlin]
GreetResponse(
  message = "Hello $name, from Micronaut",
  time = ZonedDateTime.now()
)
\end{lstlisting}

\end{enumerate}

Full example

File: \texttt{src/main/kotlin/demo/GreetingController.kt}
\begin{lstlisting}[language=Kotlin]
package demo

import io.micronaut.http.MediaType
import io.micronaut.http.annotation.Controller
import io.micronaut.http.annotation.Get
import io.micronaut.http.annotation.Produces
import java.time.ZonedDateTime

@Controller("/greeting")
class GreetingController {
  @Get("/{name}")
  @Produces(MediaType.APPLICATION_JSON)
  fun greet(name: String): GreetResponse =
    GreetResponse(
      message = "Hello $name, from Micronaut",
      time = ZonedDateTime.now()
    )
}
\end{lstlisting}

\item Build the application

\begin{lstlisting}[language=bash]
$ ./gradlew clean build
\end{lstlisting}

The tests should now pass

\begin{lstlisting}[language=bash]
...
BUILD SUCCESSFUL in 13s
18 actionable tasks: 18 executed\end{lstlisting}

\end{enumerate}



