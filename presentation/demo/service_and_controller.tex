\section{Use the Service from the Controller}\label{sec:service_and_controller}


\begin{enumerate}

\item Create an interface, named \texttt{GreetingService}, and make the \texttt{DefaultGreetingService} implement it.  This interface will be used later on to mock the service while testing the controller.

File: \texttt{src/main/kotlin/demo/GreetingService.kt}
\begin{lstlisting}[language=Kotlin]
package demo

interface GreetingService {
  fun greet(name: String): GreetResponse;
}
\end{lstlisting}

Implement the interface.

File: \texttt{src/main/kotlin/demo/DefaultGreetingService.kt}
\begin{lstlisting}[language=Kotlin]
package demo

import java.time.ZonedDateTime

class DefaultGreetingService: GreetingService {
  override fun greet(name: String): GreetResponse =
    GreetResponse(
      message = "Hello $name, from Micronaut",
      time = ZonedDateTime.now()
    )
}
\end{lstlisting}

Run the tests

\begin{lstlisting}[language=bash]
./gradlew clean build
\end{lstlisting}

The tests should all pass

\begin{lstlisting}[language=bash]
...
  > Task :test

demo.DefaultGreetingServiceTest > should return the greeting message for the given name PASSED

demo.GreetingControllerTest > should return the greeting message returned by the greeting service PASSED

...
BUILD SUCCESSFUL in 10s
18 actionable tasks: 18 executed
\end{lstlisting}

\item Custom Matcher

The \texttt{shouldBe} assersion will not work with dates with different time zones.  The following will fail due to different time zones.

\begin{lstlisting}[language=Kotlin]
val a = ZonedDateTime.now(ZoneId.of("Europe/Berlin"))
val b = a.withZoneSameInstant(ZoneOffset.UTC)
a shouldBe b
\end{lstlisting}

The dates can be compared as follow

\begin{lstlisting}[language=Kotlin]
a.isEqual(b) shouldBe true
\end{lstlisting}

The above will work, but will produce an unuseful message.

Create a custom

File: \texttt{src/test/kotlin/demo/CustomMatcher.kt}
\begin{lstlisting}[language=Kotlin]
package demo

import io.kotlintest.Matcher
import io.kotlintest.MatcherResult
import io.kotlintest.should
import io.kotlintest.shouldNot
import java.time.ZonedDateTime

infix fun ZonedDateTime.shouldNotHaveSameInstantAs(other: ZonedDateTime) =
    this shouldNot haveSameInstantAs(other)

infix fun ZonedDateTime.shouldHaveSameInstantAs(other: ZonedDateTime) =
    this should haveSameInstantAs(other)

fun haveSameInstantAs(other: ZonedDateTime) = object : Matcher<ZonedDateTime> {
  override fun test(value: ZonedDateTime): MatcherResult =
    MatcherResult(
      passed = value.isEqual(other),
      failureMessage = "$value should have same instance as $other",
      negatedFailureMessage = "$value should not have same instance as $other"
    )
}
\end{lstlisting}


\item Mock the service

File: \texttt{src/test/kotlin/demo/GreetingControllerTest.kt}
\begin{lstlisting}[language=Kotlin]
package demo

import io.kotlintest.shouldBe
import io.kotlintest.specs.StringSpec
import io.micronaut.http.client.RxHttpClient
import io.micronaut.http.client.annotation.Client
import io.micronaut.test.annotation.MicronautTest
import io.micronaut.test.annotation.MockBean
import io.micronaut.test.extensions.kotlintest.MicronautKotlinTestExtension.getMock
import io.mockk.confirmVerified
import io.mockk.every
import io.mockk.mockk
import io.mockk.verify
import java.time.ZonedDateTime

@MicronautTest
class GreetingControllerTest(
  private val service: GreetingService,
  @Client("/greeting") private val client: RxHttpClient
) : StringSpec({
  "should return the greeting message returned by the greeting service" {
    val mock = getMock(service)

    val name = "Albert"
    val greeting = GreetResponse(
      message = "Hello $name, from Micronaut",
      time = ZonedDateTime.now()
    )
    every { mock.greet(name) } returns greeting

    val result = client.toBlocking().retrieve("/$name", GreetResponse::class.java)
    result.message shouldBe greeting.message
    result.time shouldHaveSameInstantAs greeting.time

    verify(exactly = 1) { mock.greet(name) }

    /* Invoked by Micronaut */
    verify(exactly = 2) { mock.hashCode() }
    confirmVerified(mock)
  }
}) {
  @MockBean(GreetingService::class)
  fun greetingService(): GreetingService =
    mockk()
}
\end{lstlisting}

\item Run the tests

\begin{lstlisting}[language=bash]
./gradlew clean build
\end{lstlisting}

The \texttt{GreetingControllerTest} test will fail as expected

\begin{lstlisting}[language=bash]
...
> Task :test

demo.DefaultGreetingServiceTest > should return the greeting message for the given name PASSED

demo.GreetingControllerTest > should return the greeting message returned by the greeting service FAILED
    org.opentest4j.AssertionFailedError at GreetingControllerTest.kt:34

2 tests completed, 1 failed

> Task :test FAILED

FAILURE: Build failed with an exception.

...

BUILD FAILED in 17s
18 actionable tasks: 18 executed
\end{lstlisting}

The test will fail as the date have a different instant value.

\begin{lstlisting}[language=bash]
java.lang.AssertionError: 2020-04-28T05:56:39.093052Z[UTC] should have same instance as 2020-04-28T07:56:38.680259+02:00[Europe/Berlin]
  at demo.CustomMatcherKt.shouldHaveSameInstantAs(CustomMatcher.kt:13)
  at demo.GreetingControllerTest$1$1.invokeSuspend(GreetingControllerTest.kt:33)
\end{lstlisting}

The controller is not making use of the service yet and it is creating the instance of \texttt{GreetResponse}.

\item Use the service in the controller

File: \texttt{src/main/kotlin/demo/GreetingController.kt}
\begin{lstlisting}[language=Kotlin]
package demo

import io.micronaut.http.MediaType
import io.micronaut.http.annotation.Controller
import io.micronaut.http.annotation.Get
import io.micronaut.http.annotation.Produces

@Controller("/greeting")
class GreetingController(
  private val service: GreetingService
) {
  @Get("/{name}")
  @Produces(MediaType.APPLICATION_JSON)
  fun greet(name: String): GreetResponse =
    service.greet(name)
}
\end{lstlisting}

\item Run the tests again

\begin{lstlisting}[language=bash]
./gradlew clean build
\end{lstlisting}

All tests should pass.

\begin{lstlisting}[language=bash]
...
> Task :test

demo.DefaultGreetingServiceTest > should return the greeting message for the given name PASSED

demo.GreetingControllerTest > should return the greeting message returned by the greeting service PASSED

...

BUILD SUCCESSFUL in 12s
18 actionable tasks: 18 executed
\end{lstlisting}

\item Run the application

\begin{lstlisting}[language=bash]
$ java -jar build/libs/demo*-all.jar
19:09:29.709 [main] INFO  io.micronaut.runtime.Micronaut - Startup completed in 924ms. Server Running: http://localhost:8080
\end{lstlisting}

Try the service

\begin{lstlisting}[language=bash]
$ curl http://localhost:8080/greeting/Albert
{"message":"Internal Server Error: Failed to inject value for parameter [service] of class: demo.GreetingController\n\nMessage: No bean of type [demo.GreetingService] exists. Make sure the bean is not disabled by bean requirements (enable trace logging for 'io.micronaut.context.condition' to check) and if the bean is enabled then ensure the class is declared a bean and annotation processing is enabled (for Java and Kotlin the 'micronaut-inject-java' dependency should be configured as an annotation processor).\nPath Taken: new GreetingController([GreetingService service])"}
\end{lstlisting}

Micronaut is not able to locat an implementation for the \texttt{GreetingService} service.

\item Define the implementation for the \texttt{GreetingService} service.

File: \texttt{src/main/kotlin/demo/DefaultGreetingService.kt}
\begin{lstlisting}[language=Kotlin]
package demo

import java.time.ZonedDateTime
import javax.inject.Singleton

@Singleton
class DefaultGreetingService : GreetingService {
  override fun greet(name: String): GreetResponse =
    GreetResponse(
      message = "Hello $name, from Micronaut",
      time = ZonedDateTime.now()
    )
}
\end{lstlisting}

\end{enumerate}


