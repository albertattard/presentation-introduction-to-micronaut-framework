\section{Service}\label{sec:service}

Create a service that will be responsible from creating the response that will be returned by the controller.

\begin{enumerate}

\item Create the \textbf{skeleton} service class

File: \texttt{src/main/kotlin/demo/DefaultGreetingService.kt}
\begin{lstlisting}[language=Kotlin]
package demo

 class DefaultGreetingService {
  fun greet(name:String): GreetResponse {
    TODO("Coming soon...")
  }
}
\end{lstlisting}

The \href{https://kotlinlang.org/api/latest/jvm/stdlib/kotlin/-t-o-d-o.html}{\texttt{TODO}} will enable the service to compile yet it will throw \href{https://kotlinlang.org/api/latest/jvm/stdlib/kotlin/-not-implemented-error/}{\texttt{kotlin.NotImplementedError}} when invoked.

\item Create the test

File: \texttt{src/test/kotlin/demo/DefaultGreetingServiceTest.kt}
\begin{lstlisting}[language=Kotlin]
package demo

import io.kotlintest.shouldBe
import io.kotlintest.specs.StringSpec
import io.micronaut.test.annotation.MicronautTest

@MicronautTest
class DefaultGreetingServiceTest(
) : StringSpec({
  "should return the greeting message for the given name" {
    val service = DefaultGreetingService()
    val greeting = service.greet("Albert")
    greeting.message shouldBe "Hello Albert"
  }
})
\end{lstlisting}

\item Run the tests

\begin{lstlisting}[language=bash]
$ ./gradlew clean build
\end{lstlisting}

The test will fail

\begin{lstlisting}[language=bash]
...
> Task :test

demo.DefaultGreetingServiceTest > should return the greeting message for the given name FAILED
    kotlin.NotImplementedError at DefaultGreetingServiceTest.kt:12

demo.GreetingControllerTest > should return the greeting message returned by the greeting service PASSED

2 tests completed, 1 failed

> Task :test FAILED
...
\end{lstlisting}

\item Implement the service

\begin{lstlisting}[language=Kotlin]
package demo

import java.time.ZonedDateTime

class DefaultGreetingService {
  fun greet(name: String): GreetResponse =
    GreetResponse(
      message = "Hello $name",
      time = ZonedDateTime.now()
    )
}
\end{lstlisting}

\item Run the tests again

\begin{lstlisting}[language=bash]
$ ./gradlew clean build
\end{lstlisting}

This time, both tests should pass.

\begin{lstlisting}[language=bash]
...
> Task :test

demo.DefaultGreetingServiceTest > should return the greeting message for the given name PASSED

demo.GreetingControllerTest > should return the greeting message returned by the greeting service PASSED

...

BUILD SUCCESSFUL in 10s
18 actionable tasks: 18 executed
\end{lstlisting}

\item Create an interface, named \texttt{GreetingService}, and make the \texttt{DefaultGreetingService} implement it.  This interface will be used later on to mock the service while testing the controller.

File: \texttt{src/main/kotlin/demo/GreetingService.kt}
\begin{lstlisting}[language=Kotlin]
package demo

interface GreetingService {
  fun greet(name: String): GreetResponse;
}
\end{lstlisting}

Implement the interface.


File: \texttt{src/main/kotlin/demo/DefaultGreetingService.kt}
\begin{lstlisting}[language=Kotlin]
package demo

import java.time.ZonedDateTime

class DefaultGreetingService: GreetingService {
  override fun greet(name: String): GreetResponse =
    GreetResponse(
      message = "Hello $name",
      time = ZonedDateTime.now()
    )
}
\end{lstlisting}

Run the tests

\begin{lstlisting}[language=bash]
./gradlew clean build
\end{lstlisting}

The tests should all pass

\begin{lstlisting}[language=bash]
...
  > Task :test

demo.DefaultGreetingServiceTest > should return the greeting message for the given name PASSED

demo.GreetingControllerTest > should return the greeting message returned by the greeting service PASSED

...
BUILD SUCCESSFUL in 10s
18 actionable tasks: 18 executed
\end{lstlisting}



\begin{lstlisting}[language=Kotlin]
package demo

import io.kotlintest.shouldBe
import io.kotlintest.specs.StringSpec
import io.micronaut.http.client.RxHttpClient
import io.micronaut.http.client.annotation.Client
import io.micronaut.test.annotation.MicronautTest

@MicronautTest
class GreetingControllerTest(
  private val service: GreetingService,
  @Client("/greeting") private val client: RxHttpClient
) : StringSpec({
  "should return the greeting message returned by the greeting service" {
    val mock = getMock(service)

    val greeting = GreetingResponse("Hello $name")
    every { mock.greet(name) } returns greeting

    val result = client.toBlocking().retrieve("/$name", GreetResponse::class.java)
    result.message shouldBe "Hello from Micronaut"
  }
}) {
  @MockBean(GreetingService::class)
  fun greetingService(): GreetingService =
    mockk()
}
\end{lstlisting}

\end{enumerate}


