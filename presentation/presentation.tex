\documentclass{beamer}

\usepackage[utf8]{inputenc}
\usepackage[T1]{fontenc}
\usepackage[english]{babel}
\usepackage{setspace}
\usepackage{color}
\usepackage{listings}

\usetheme[progressbar=frametitle]{metropolis}
\setbeamertemplate{frame numbering}[fraction]
\useoutertheme{metropolis}
\useinnertheme{metropolis}
\usefonttheme{metropolis}
\usecolortheme{spruce}
\setbeamercolor{background canvas}{bg=white}

\title[Micronaut]{Introduction to Micronaut Framework}
\subtitle{A modern, JVM-based, full-stack framework for building modular, easily testable microservice and serverless applications.}
\author{\texorpdfstring{Albert Attard\newline\url{albert.attard@thoughtworks.com}}{Albert Attard}}
\institute{\large \href{https://thoughtworks.com}{\textbf{ThoughtWorks}.com}}
\date{}

\begin{document}
\metroset{block=fill}

    \begin{frame}
        \titlepage
    \end{frame}


    \begin{frame}[t]{Title Introduction}
        Hello 1
    \end{frame}

    \begin{frame}[t]{using Blocks}
        \begin{block}{Micronaut}
        \textbf{Micronaut} is a modern, JVM-based, full-stack framework for building modular, easily testable microservice and serverless applications
        \end{block}
    \end{frame}

    \begin{frame}[t,fragile]{Code Example}
        Example
        \begin{lstlisting}[language=Java]
// this is a simple code listing
println("hello kotlin from latex")
        \end{lstlisting}
    \end{frame}
\end{document}
