

\begin{frame}[t]{using Blocks}
  \begin{block}{Micronaut}
    \textbf{Micronaut} is a modern, JVM-based, full-stack framework for building modular, easily testable microservice and serverless applications
  \end{block}
\end{frame}


\begin{frame}[t]{Copy and Paste}

  The term "\textit{micro web services}" was coined by \href{https://www.linkedin.com/in/pjrodgers/}{Dr. Peter Rodgers} during the Web Services Edge conference in 2005. (\href{https://web.archive.org/web/20180520124343/http://www.cloudcomputingexpo.com/node/80883}{reference})


  Netflix and Amazon were among the early adopters of Microservices


  The load of an application may vary, depending on the current usage.

  Take Netflix for example. Many subscribers will be watching the new episodes of a highly anticipated show as soon as these are released. This load will go away once the hype dies out

  The ability to adjust to the load is called \textbf{elasticity}

  An application is said to be elastic if the application can grow and shrink according to the load.

  Elasticity is the ability

  "In cloud computing, elasticity is defined as "the degree to which a system is able to adapt to workload changes by provisioning and de-provisioning resources in an autonomic manner, such that at each point in time the available resources match the current demand as closely as possible".[1][2] Elasticity is a defining characteristic that differentiates cloud computing from previously proposed computing paradigms, such as grid computing. The dynamic adaptation of capacity, e.g., by altering the use of computing resources, to meet a varying workload is called "elastic computing".[3][4]"

  Microservices and Serverless, for our context, are small applications that scale up and down as required ()

  When facing more load, new instances of our applications are quickly started and stopped when the loads is reduced

  Require fast startup

\end{frame}


\begin{frame}[t]{Spring Boot}
  Spring Boot is a sold framework that leverages the Spring ecosystem and enables fast development cycles and minimise boilerplate

  It simplifies bootstrapping and development of new Spring based applications by providing opinionated defaults.

  Spring Boot provides a vast number of libraries that take advantage of it and make is easy to adopt (\href{https://docs.spring.io/spring-boot/docs/2.0.0.M5/reference/html/boot-features-developing-auto-configuration.html}{using the auto-configuration}).

  \vspace{16pt}
  \textbf{So why don't we just use Spring Boot?}
\end{frame}




\begin{frame}[t]{Another Framework?}
  \textbf{Spring Boot} dominates the market, according to \href{https://www.jrebel.com/sites/rebel/files/pdfs/ebook-jrebel-java-productivity-report.pdf}{JRebel's 2020 Java Developer Productivity Report}


  \vspace{16pt}
  \begin{center}
    \begin{tikzpicture}[scale=0.75]
      \pie [rotate = 0, color={green!5, green!10}]
      {82.70/Spring Boot,
      17.3/Others}
    \end{tikzpicture}

  \end{center}
\end{frame}

